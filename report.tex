% Options for packages loaded elsewhere
\PassOptionsToPackage{unicode}{hyperref}
\PassOptionsToPackage{hyphens}{url}
\PassOptionsToPackage{dvipsnames,svgnames,x11names}{xcolor}
%
\documentclass[
  a4paper]{article}

\usepackage{amsmath,amssymb}
\usepackage{iftex}
\ifPDFTeX
  \usepackage[T1]{fontenc}
  \usepackage[utf8]{inputenc}
  \usepackage{textcomp} % provide euro and other symbols
\else % if luatex or xetex
  \usepackage{unicode-math}
  \defaultfontfeatures{Scale=MatchLowercase}
  \defaultfontfeatures[\rmfamily]{Ligatures=TeX,Scale=1}
\fi
\usepackage{lmodern}
\ifPDFTeX\else  
    % xetex/luatex font selection
\fi
% Use upquote if available, for straight quotes in verbatim environments
\IfFileExists{upquote.sty}{\usepackage{upquote}}{}
\IfFileExists{microtype.sty}{% use microtype if available
  \usepackage[]{microtype}
  \UseMicrotypeSet[protrusion]{basicmath} % disable protrusion for tt fonts
}{}
\makeatletter
\@ifundefined{KOMAClassName}{% if non-KOMA class
  \IfFileExists{parskip.sty}{%
    \usepackage{parskip}
  }{% else
    \setlength{\parindent}{0pt}
    \setlength{\parskip}{6pt plus 2pt minus 1pt}}
}{% if KOMA class
  \KOMAoptions{parskip=half}}
\makeatother
\usepackage{xcolor}
\usepackage[margin=1in]{geometry}
\setlength{\emergencystretch}{3em} % prevent overfull lines
\setcounter{secnumdepth}{5}
% Make \paragraph and \subparagraph free-standing
\ifx\paragraph\undefined\else
  \let\oldparagraph\paragraph
  \renewcommand{\paragraph}[1]{\oldparagraph{#1}\mbox{}}
\fi
\ifx\subparagraph\undefined\else
  \let\oldsubparagraph\subparagraph
  \renewcommand{\subparagraph}[1]{\oldsubparagraph{#1}\mbox{}}
\fi


\providecommand{\tightlist}{%
  \setlength{\itemsep}{0pt}\setlength{\parskip}{0pt}}\usepackage{longtable,booktabs,array}
\usepackage{calc} % for calculating minipage widths
% Correct order of tables after \paragraph or \subparagraph
\usepackage{etoolbox}
\makeatletter
\patchcmd\longtable{\par}{\if@noskipsec\mbox{}\fi\par}{}{}
\makeatother
% Allow footnotes in longtable head/foot
\IfFileExists{footnotehyper.sty}{\usepackage{footnotehyper}}{\usepackage{footnote}}
\makesavenoteenv{longtable}
\usepackage{graphicx}
\makeatletter
\def\maxwidth{\ifdim\Gin@nat@width>\linewidth\linewidth\else\Gin@nat@width\fi}
\def\maxheight{\ifdim\Gin@nat@height>\textheight\textheight\else\Gin@nat@height\fi}
\makeatother
% Scale images if necessary, so that they will not overflow the page
% margins by default, and it is still possible to overwrite the defaults
% using explicit options in \includegraphics[width, height, ...]{}
\setkeys{Gin}{width=\maxwidth,height=\maxheight,keepaspectratio}
% Set default figure placement to htbp
\makeatletter
\def\fps@figure{htbp}
\makeatother

\makeatletter
\makeatother
\makeatletter
\makeatother
\makeatletter
\@ifpackageloaded{caption}{}{\usepackage{caption}}
\AtBeginDocument{%
\ifdefined\contentsname
  \renewcommand*\contentsname{Table of contents}
\else
  \newcommand\contentsname{Table of contents}
\fi
\ifdefined\listfigurename
  \renewcommand*\listfigurename{List of Figures}
\else
  \newcommand\listfigurename{List of Figures}
\fi
\ifdefined\listtablename
  \renewcommand*\listtablename{List of Tables}
\else
  \newcommand\listtablename{List of Tables}
\fi
\ifdefined\figurename
  \renewcommand*\figurename{Figure}
\else
  \newcommand\figurename{Figure}
\fi
\ifdefined\tablename
  \renewcommand*\tablename{Table}
\else
  \newcommand\tablename{Table}
\fi
}
\@ifpackageloaded{float}{}{\usepackage{float}}
\floatstyle{ruled}
\@ifundefined{c@chapter}{\newfloat{codelisting}{h}{lop}}{\newfloat{codelisting}{h}{lop}[chapter]}
\floatname{codelisting}{Listing}
\newcommand*\listoflistings{\listof{codelisting}{List of Listings}}
\makeatother
\makeatletter
\@ifpackageloaded{caption}{}{\usepackage{caption}}
\@ifpackageloaded{subcaption}{}{\usepackage{subcaption}}
\makeatother
\makeatletter
\@ifpackageloaded{tcolorbox}{}{\usepackage[skins,breakable]{tcolorbox}}
\makeatother
\makeatletter
\@ifundefined{shadecolor}{\definecolor{shadecolor}{rgb}{.97, .97, .97}}
\makeatother
\makeatletter
\makeatother
\makeatletter
\makeatother
\ifLuaTeX
  \usepackage{selnolig}  % disable illegal ligatures
\fi
\usepackage[]{biblatex}
\addbibresource{bibliography.bib}
\nocite{*}
\IfFileExists{bookmark.sty}{\usepackage{bookmark}}{\usepackage{hyperref}}
\IfFileExists{xurl.sty}{\usepackage{xurl}}{} % add URL line breaks if available
\urlstyle{same} % disable monospaced font for URLs
\hypersetup{
  pdftitle={Bitcoin sentiment analysis},
  pdfauthor={Ambroise Thibault, Faune Blanchard, Maxime Lorenzo},
  colorlinks=true,
  linkcolor={blue},
  filecolor={Maroon},
  citecolor={Blue},
  urlcolor={Blue},
  pdfcreator={LaTeX via pandoc}}

\title{Bitcoin sentiment analysis}
\author{Ambroise Thibault, Faune Blanchard, Maxime Lorenzo}
\date{}

\begin{document}
\maketitle
\ifdefined\Shaded\renewenvironment{Shaded}{\begin{tcolorbox}[boxrule=0pt, breakable, enhanced, interior hidden, frame hidden, borderline west={3pt}{0pt}{shadecolor}, sharp corners]}{\end{tcolorbox}}\fi

\renewcommand*\contentsname{Table of contents}
{
\hypersetup{linkcolor=}
\setcounter{tocdepth}{4}
\tableofcontents
}
\hypertarget{introduction}{%
\section{Introduction}\label{introduction}}

Write with chat

\hypertarget{generating-sentiment-score}{%
\section{Generating sentiment score}\label{generating-sentiment-score}}

\hypertarget{presentation-of-the-database}{%
\subsection{Presentation of the
database}\label{presentation-of-the-database}}

We collected a dataset from GitHub entitled
\href{https://github.com/soheilrahsaz/cryptoNewsDataset}{Crypto News
Dataset}. It aggregates daily headlines from a variety of crypto news
websites. Each headline is labeled with the cryptocurrencies to which it
refers. They are also timestamped down to the second the news was posted
on the site. We also have the number of likes, dislikes, comments and
others, although these are often 0, as well as the link to the article.
In this project, we will only be using the raw headlines in order to
generate the scores that we need.

Here is a quick view of the pre processed dataset with our function
``import\_data'' in the ``prompts'' module. This function reads the csv
files from the GitHub repository linked above, then it removes unwanted
columns and formats the date so that we only have the day at which it
was posted. We can see that there can be multiple headlines in one day,
or no headlines. This will cause issues that we will explain later on in
the project.

The function also writes the prompts that we will later feed to the
Large Language Model in order to properly generate the scores, which we
will explain in the following section.

\hypertarget{generating-prompts}{%
\subsection{Generating prompts}\label{generating-prompts}}

For each headline, we generate a prompt which asks the AI its beliefs
about the evolution of Bitcoin returns based on this headline. The
prompt we decided to use is the exact prompt presented in
\textcite{leland_bybee_2025} . The exact prompt reads :

\begin{verbatim}
Here is a piece of news:
"%s"
Do you think this news will increase or decrease BTC?
Write your answer as: 
{increase/decrease/uncertain}:
{confidence (0-1)}:
{magnitude of increase/decrease (0-1)}:
{explanation (less than 25 words)}
\end{verbatim}

Where ``\%s'' is replaced by the headline. After editing this prompt, we
notice that removing part of this prompt changes the results drastically
(from ``increase'' to ``decrease'' for the same headline). Therefore, we
decided to keep the prompt from the paper as is, even though we will
only use the results for ``increase'' or ``decrease'' for this project.
This prompt has been adapted from traditional retail investor surveys or
surveys of CFOs that are used to produce a sentiment indicator. However,
these are costly and slow. As presented in the paper, using AI models to
replace human answers mimics the same results on a much bigger and
faster scale. We will also try to edit this prompt by adding context:
since Bitcoin has mostly been popular among retail investors, we can add
a simple sentence (e.g.~``You are a retail investor'') at the beginning
of this prompt, to see if this yields better results. We will conclude
on this in the end of the project.

More information about the construction on this prompt can be found in
\textcite{leland_bybee_2025} .

\hypertarget{scores-methodology}{%
\subsection{Scores methodology}\label{scores-methodology}}

\hypertarget{comparison-with-btc-returns}{%
\section{Comparison with BTC
returns}\label{comparison-with-btc-returns}}

\hypertarget{extracting-btc-returns}{%
\subsection{Extracting BTC returns}\label{extracting-btc-returns}}

\hypertarget{comparing-with-scores}{%
\subsection{Comparing with scores}\label{comparing-with-scores}}

\hypertarget{initial-results}{%
\subsubsection{Initial results}\label{initial-results}}

\hypertarget{problems-with-our-model}{%
\subsubsection{Problems with our model}\label{problems-with-our-model}}

\hypertarget{interpretation}{%
\subsubsection{Interpretation}\label{interpretation}}


\printbibliography[title=Conclusion]


\end{document}
